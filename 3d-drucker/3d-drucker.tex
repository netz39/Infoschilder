\documentclass{article}

\usepackage{helvet}
\usepackage[utf8]{inputenc}
\usepackage[T1]{fontenc}
\usepackage[ngerman]{babel}

\usepackage{svg}
\usepackage{graphicx}
\usepackage{geometry}
\usepackage{setspace}
\usepackage{caption}

% Helvetica als Standardschriftart
\renewcommand{\familydefault}{\sfdefault}

% Seitenränder einstellen
\geometry{
	a4paper,
	left=25mm,
	right=25mm,
	top=10mm,
	bottom=10mm
}

% Zeilenabstand einstellen
\setstretch{0.5} % Hier den gewünschten Zeilenabstand einstellen

% Automatische Nummerierung von Bildunterschriften deaktivieren
\captionsetup{labelformat=empty}

% Seitennummerierung deaktivieren
\pagestyle{empty}


\begin{document}
	
	\noindent
	\begin{minipage}{0.25\textwidth}
		\section*{3D-Drucker}
	\end{minipage}
	\hfill
	\begin{minipage}{0.1\textwidth}
		\includegraphics[width=\textwidth]{3d-printing.png}
	\end{minipage}
	\hspace{2em}	
	\begin{minipage}{0.1\textwidth}
			\includesvg[inkscapelatex=false, width=\textwidth]{../Symbole/netz39-logo-final.svg}	
	\end{minipage}%
	
	\subsection*{Wichtige Informationen}
	\begin{itemize}
		\item \textbf{Modelle:}
		\begin{itemize}
			\item Delta-Drucker \texttt{AnyCubic Kossel Linear Plus} (230 × 230 × 300 mm)
			\item \texttt{Tevo Tarantula} (220 x 280 x 186mm)
		\end{itemize}
		\item \textbf{Druckerserver:} Repetier-Server
	\end{itemize}
	
		\noindent\dotfill
	\subsection*{Ansprechpartner}
	Discord-Channel: \colorbox{gray!30}{\texttt{\#3d-druck}}
	\vspace{0.5em}\\
	
	\noindent\dotfill
	\subsection*{Workflow}
	Detaillierte Infos sind auf der\textbf{ Wiki-Seite} zu finden! (Siehe QR-Code)
	\begin{itemize}
		\item Verteilerleiste einschalten, ggf. Stromschalter am Drucker einlegen
		\item GCode vorbereiten
		\item GCode über Repetier-Server an Drucker schicken
		\item Druck überwachen
	\end{itemize}	
	
	\noindent\dotfill
	\subsection*{Links}
	\begin{minipage}{0.4\textwidth}
		\centering
		Wiki-Seite:\\[0.5em]
		\includesvg[inkscapelatex=false, width=0.5\linewidth]{wiki-3d-drucker.svg}\\[0.5em]
		https://wiki.netz39.de/internal:inventory:tools:3d\_printer\\
		[0.5em]
	\end{minipage}
	\hfill
	\begin{minipage}{0.4\textwidth}
		\centering
		Repetier-Server:\\[0.5em]
		\includesvg[inkscapelatex=false, width=0.5\linewidth]{repetier-server.svg}\\[0.5em]
		repetier-server.local \\
	\end{minipage}\\
	
	\noindent\dotfill\\
	
	\noindent \begin{minipage}{0.4\textwidth}
		\centering
		Throubleshooting von Make:\\[0.5em]
		\includesvg[inkscapelatex=false, width=0.5\linewidth]{make-throubleshooting.svg}\\[0.5em]
		https://makezine.com/article/digital-fabrication/3d-printing-workshop/the-best-3d-printing-troubleshooting-guide/\\
		[0.5em]
	\end{minipage}
	\hfill
	\begin{minipage}{0.4\textwidth}
		\centering
		Throubleshooting von Simplify3D:\\[0.5em]
		\includesvg[inkscapelatex=false, width=0.5\linewidth]{simplify3d-throubleshooting.svg}\\[0.5em]
		https://www.simplify3d.com/resources/print-quality-troubleshooting/ \\
	\end{minipage}
	\vspace{1em}
	
	\noindent\fbox{
	\begin{minipage}{\textwidth}
		\centering	
		\subsection*{Arbeitsschutz}
		\begin{minipage}{0.33\textwidth}
			\centering
			\includesvg[inkscapelatex=false, width=0.6\linewidth]{../Symbole/DIN_4844-2_D-M018.svg}
			\captionof{figure}{Einweisung nötig!}
		\end{minipage}%
		\begin{minipage}{0.33\textwidth}
			\centering
			\includesvg[inkscapelatex=false, width=0.7\linewidth]{../Symbole/D-W027_Warnung_vor_Handverletzungen.svg}
			\captionof{figure}{\mbox{Vorsicht vor heißem Extruder/Heizbett!}}
		\end{minipage}
	\end{minipage}
	}
	\vfill
	\begin{center}
		\color{darkgray}
		Das 3D-Drucker Icon ist von Flaticon.com
	\end{center}
	
\end{document}