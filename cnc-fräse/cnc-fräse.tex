\documentclass{article}
\usepackage{helvet}
\usepackage{svg}
\usepackage{graphicx}
\usepackage{geometry}
\usepackage{setspace}
\usepackage{caption}

% Helvetica als Standardschriftart
\renewcommand{\familydefault}{\sfdefault}

% Seitenränder einstellen
\geometry{
	a4paper,
	left=25mm,
	right=25mm,
	top=10mm,
	bottom=10mm
}

% Zeilenabstand einstellen
\setstretch{0.5} % Hier den gewünschten Zeilenabstand einstellen

% Automatische Nummerierung von Bildunterschriften deaktivieren
\captionsetup{labelformat=empty}

% Seitennummerierung deaktivieren
\pagestyle{empty}


\begin{document}
	
	\noindent
	\begin{minipage}{0.25\textwidth}
		\section*{CNC-Fräse}
	\end{minipage}
	\begin{minipage}{0.75\textwidth}
		\begin{flushright}
			\includesvg[inkscapelatex=false, width=50pt]{../Symbole/netz39-logo-final.svg}
		\end{flushright}	
	\end{minipage}%
	
	\subsection*{Wichtige Informationen}
	\begin{itemize}
		\item \textbf{Modell:} Stepcraft 840/2
		\item \textbf{Arbeitsbereich:} 600mm x 840mm x 140 mm
		\item \textbf{Steuerung:} Arduino mit GBRL-Shield
		\item \textbf{Spindelgeschwindigkeit:} max 20.000 U/min
	\end{itemize}
	
	\subsection*{Workflow}
	Detaillierte Infos sind auf der\textbf{ Wiki-Seite} zu finden!
	\begin{itemize}
		\item Fräse über Staubsauger (Werkzeugmodus) einschalten $\rightarrow$ Licht geht an, Kompressor knattert leise
		\item NUC hochfahren, mit Estlecam den GCode vorbereiten
		\item mit Candle homen und Nullpunkt einstellen
		\item Werkstück auf Opferplatte befestigen
		\item Fräser einspannen
		\item sicherstellen, dass der Motorcontroller für die Spindel eingeschaltet ist
		\item mit Candle Fräsvorgang starten
	\end{itemize}
	
	\subsection*{Troubleshooting}
	\begin{itemize}
		\item \textbf{Problem:} Spindel läuft nicht an
		\begin{itemize}
			\item \textit{Lösung:}
			\begin{itemize}
				\item Fräsmotor-Controller eingeschaltet?
				\item Staubsauger auf Werkzeugmodus?
			\end{itemize}
		\end{itemize}
	\end{itemize}
	
	\subsection*{Discord-Channel}
		\colorbox{gray!30}{\texttt{\#werkzeuge}}
	
	\subsection*{Links}
	\begin{center}
		
		\renewcommand{\arraystretch}{2.0}
		\begin{tabular}{|c|c|}
			\hline
			Wiki-Seite: & Fräse schmieren: \\
			\includesvg[inkscapelatex=false, width=0.2\linewidth]{wiki-fräse.svg} & \includesvg[inkscapelatex=false, width=0.2\linewidth]{yt-fräse-schmieren.svg} \\
			https://wiki.netz39.de/internal:inventory:tools:stepcraft & https://youtu.be/meZG8agCek4 \\
			\hline
		\end{tabular}
	\end{center}
	
	\subsection*{Arbeitsschutz}
	\begin{figure}[h]
		\centering
		\begin{minipage}{0.33\textwidth}
			\centering
			\includesvg[inkscapelatex=false, width=0.65\linewidth]{../Symbole/DIN_4844-2_D-M018.svg}
			\caption*{Einweisung nötig!}
		\end{minipage}%
		\begin{minipage}{0.33\textwidth}
			\centering
			\includesvg[inkscapelatex=false, width=0.65\linewidth]{../Symbole/DIN_4844-2_D-M003.svg}
			\caption*{Gehörschutz tragen!}
		\end{minipage}
		\begin{minipage}{0.33\textwidth}
			\centering
			\includesvg[inkscapelatex=false, width=0.65\linewidth]{../Symbole/D-W027_Warnung_vor_Handverletzungen.svg}
			\caption*{\mbox{Vorsicht vor rotierendes Werkzeug}, keine Handschuhe tragen!}
		\end{minipage}
	\end{figure}
	
\end{document}
