\documentclass{article}

\usepackage{helvet}
\usepackage[utf8]{inputenc}
\usepackage[T1]{fontenc}
\usepackage[ngerman]{babel}

\usepackage{svg}
\usepackage{graphicx}
\usepackage{geometry}
\usepackage{setspace}
\usepackage{caption}

% Helvetica als Standardschriftart
\renewcommand{\familydefault}{\sfdefault}

% Seitenränder einstellen
\geometry{
	a4paper,
	left=25mm,
	right=25mm,
	top=10mm,
	bottom=10mm
}

% Zeilenabstand einstellen
\setstretch{0.5} % Hier den gewünschten Zeilenabstand einstellen

% Automatische Nummerierung von Bildunterschriften deaktivieren
\captionsetup{labelformat=empty}

% Seitennummerierung deaktivieren
\pagestyle{empty}


\begin{document}
	
	\noindent
	\begin{minipage}{0.25\textwidth}
		\section*{CNC-Fräse}
	\end{minipage}
	\begin{minipage}{0.75\textwidth}
		\begin{flushright}
			\includesvg[inkscapelatex=false, width=50pt]{../Symbole/netz39-logo-final.svg}
		\end{flushright}	
	\end{minipage}%
	
	\subsection*{Wichtige Informationen}
	\begin{itemize}
		\item \textbf{Modell:} Stepcraft 840/2
		\item \textbf{Arbeitsbereich:} 600mm x 840mm x 140 mm
		\item \textbf{Steuerung:} Arduino mit GBRL-Shield
		\item \textbf{Spindelgeschwindigkeit:} max 20.000 U/min
	\end{itemize}
	
	\noindent\dotfill
	\subsection*{Workflow}
	Detaillierte Infos sind auf der\textbf{ Wiki-Seite} zu finden! (Siehe QR-Code)
	\begin{itemize}
		\item Fräse über Staubsauger (Werkzeugmodus) einschalten $\rightarrow$ Licht geht an, Kompressor knattert leise
		\item NUC hochfahren, mit \texttt{Estlecam} den GCode vorbereiten
		\item \texttt{Candle} starten, Verbindung zur Fräse herstellen, homen und Nullpunkt einstellen
		\item Werkstück auf der Opferplatte befestigen
		\item Fräser einspannen
		\item sicherstellen, dass der Motorcontroller für die Spindel eingeschaltet ist
		\item mit \texttt{Candle} GCode laden und Fräsvorgang starten
	\end{itemize}
	
	\noindent\dotfill
	\subsection*{Troubleshooting}
	\begin{itemize}
		\item \textbf{Problem:} Spindel läuft nicht an
		\begin{itemize}
			\textit{Lösung:}
			\begin{itemize}
				\item Fräsmotor-Controller einschalten
				\item Staubsauger auf Werkzeugmodus stellen
			\end{itemize}
		\end{itemize}
	\end{itemize}
	
	\noindent\dotfill
	\subsection*{Discord-Channel}
		\colorbox{gray!30}{\texttt{\#werkzeuge}}
	\vspace{1em}\\
	
	\noindent\dotfill
	\subsection*{Links}
	\begin{minipage}{0.45\textwidth}
		\centering
		Wiki-Seite:\\
		\includesvg[inkscapelatex=false, width=0.5\linewidth]{wiki-fräse.svg}\\
		https://wiki.netz39.de/internal:inventory:tools:stepcraft
	\end{minipage}
	\hfill
	\begin{minipage}{0.45\textwidth}
		\centering
		Fräse schmieren:\\
		\includesvg[inkscapelatex=false, width=0.5\linewidth]{yt-fräse-schmieren.svg} \\
		https://youtu.be/meZG8agCek4 \\
	\end{minipage}
	\vspace{1em}
	
	\noindent\fbox{
	\begin{minipage}{\textwidth}
		\centering	
		\subsection*{Arbeitsschutz}
		\begin{minipage}{0.33\textwidth}
			\centering
			\includesvg[inkscapelatex=false, width=0.65\linewidth]{../Symbole/DIN_4844-2_D-M018.svg}
			\captionof{figure}{Einweisung nötig!}
		\end{minipage}%
		\begin{minipage}{0.33\textwidth}
			\centering
			\includesvg[inkscapelatex=false, width=0.65\linewidth]{../Symbole/DIN_4844-2_D-M003.svg}
			\captionof{figure}{Gehörschutz tragen!}
		\end{minipage}
		\begin{minipage}{0.33\textwidth}
			\centering
			\includesvg[inkscapelatex=false, width=0.65\linewidth]{../Symbole/D-W027_Warnung_vor_Handverletzungen.svg}
			\captionof{figure}{\mbox{Vorsicht vor rotierendes Werkzeug}, keine Handschuhe tragen!}
		\end{minipage}
	\end{minipage}
	}
	
\end{document}