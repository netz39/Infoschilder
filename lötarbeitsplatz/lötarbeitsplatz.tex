\documentclass[a5paper]{article}

\usepackage{helvet}
\usepackage[utf8]{inputenc}
\usepackage[T1]{fontenc}
\usepackage[ngerman]{babel}

\usepackage{svg}
\usepackage{graphicx}
\usepackage{geometry}
\usepackage{setspace}
\usepackage{caption}

\usepackage[fontsize=8pt]{fontsize}

% Helvetica als Standardschriftart
\renewcommand{\familydefault}{\sfdefault}

% Seitenränder einstellen
\geometry{
	a5paper,
	left=12mm,
	right=12mm,
	top=5mm,
	bottom=5mm
}

% Zeilenabstand einstellen
\setstretch{0.5} % Hier den gewünschten Zeilenabstand einstellen

% Automatische Nummerierung von Bildunterschriften deaktivieren
\captionsetup{labelformat=empty}

% Seitennummerierung deaktivieren
\pagestyle{empty}


\begin{document}
	
	\noindent
	\begin{minipage}{0.25\textwidth}
		\section*{Lötarbeitsplatz}
	\end{minipage}
	\hfill
	\begin{minipage}{0.1\textwidth}
		\includegraphics[width=\textwidth]{soldering-iron.png}	
	\end{minipage}
	\hspace{1em}
	\begin{minipage}{0.1\textwidth}
			\includesvg[inkscapelatex=false, width=\textwidth]{../Symbole/netz39-logo-final.svg}	
	\end{minipage}%
	
	\subsection*{Wichtige Informationen}
	\begin{itemize}
		\item \textbf{Lötstation:} \texttt{JBC CD2E020}
		\item \textbf{mobile Lötkolben:} \texttt{PINECIL v1 und v2}
		\item \textbf{Entlötstation:} \texttt{PACE ST 115 Basisstation und SX-70 Kolben}
		\item \textbf{Lötdampfabsaugung:} \texttt{qubo Fume Extraction Unit}
	\end{itemize}
	
	\noindent\dotfill
	\subsection*{Ansprechpartner}
	\begin{itemize}
		\item Discord-Channel: \colorbox{gray!30}{\texttt{\#werkzeuge}}
	\end{itemize}
	
	\noindent\dotfill
	\subsection*{Workflow}
	\begin{itemize}
	    \item Stromversorgung des Arbeitsplatzes über die Verteilerleiste einschalten\\ (auf der rechten Seite links neben der Entlötstation)
	    \item ggf. Arbeitsplatzbeleuchtung aus dem Standby wecken (geht nach 45min Inaktivität aus)
	\end{itemize}
	
		
	\noindent\dotfill
	\subsection*{Links}
	\begin{center}
		\begin{minipage}{0.6\textwidth}
			\centering
			Wiki-Seite:\\[0.5em]
			\includesvg[inkscapelatex=false, width=0.5\textwidth]{wiki-lötarbeitsplatz.svg}\\[0.5em]
			https://wiki.netz39.de/internal:inventory:tools:loetarbeitsplatz
		\end{minipage}
	\end{center}
	
	\vspace{1em}
	
	\noindent\fbox{
	\begin{minipage}{\textwidth}
		\centering	
		\subsection*{Arbeitsschutz}
		\begin{minipage}{0.33\textwidth}
			\centering
			\includesvg[inkscapelatex=false, width=0.65\linewidth]{../Symbole/DIN_4844-2_D-M018.svg}
			\captionof{figure}{Einweisung nötig}
		\end{minipage}%
		\begin{minipage}{0.33\textwidth}
			\centering
			\includesvg[inkscapelatex=false, width=0.75\linewidth]{../Symbole/D-W027_Warnung_vor_Handverletzungen.svg}
			\captionof{figure}{Vorsicht vor heißen Werkzeugen}
		\end{minipage}
	\end{minipage}
	}
	\vfill
	\begin{center}
		\color{darkgray}
		Das Lötkolben-Icon ist von Flaticon.com
	\end{center}
	
\end{document}